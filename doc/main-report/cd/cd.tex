\documentclass{article}
\usepackage{../public/pgf-umlcd-f}
%xeCJK
\usepackage[BoldFont,SlantFont,CJKchecksingle]{xeCJK}
\setCJKmainfont[BoldFont=SimHei,SlantedFont=KaiTi]{SimSun}

\begin{document}
\section{Common Layer}
\begin{tikzpicture}
\begin{class}[text width=7cm]{Vec}{0,-0.3}
\attribute{- m\_X : double}
\attribute{- m\_Y : double}
\operation{+ setX(x:double) : bool}
\operation{+ setY(y:double) : bool}
\operation{+ getX() : double}
\operation{+ getY() : double}
\operation{+ getAngle() : double}
\operation{+ getMagnitude() : double}
\operation{+ rotate(center:Vec,angle:double) : void}
\end{class}
\begin{class}[text width=7cm]{Segment}{4,6.2}
\attribute{- m\_Vertex1 : Vec}
\attribute{- m\_Vertex2 : Vec}
\operation{+ getV1() : Vec}
\operation{+ getV2() : Vec}
\operation{+ setV1(v:Vec) : bool}
\operation{+ setV2(v:Vec) : bool}
\end{class}
\begin{class}[text width=7.2cm]{Poly}{-4,6.2}
\attribute{- m\_CenterPoint: Vec}
\attribute{- m\_Point : vector<Vec>}
\attribute{- m\_PointNum : int}
\attribute{- m\_area : double}
\operation{+ calcCenterPointAndArea() : void}
\operation{+ setPoly(center:Vec,pol:vector<Vec>) : bool}
\operation{+ inPoly\_Vec(v:Vec) : bool}
\operation{+ getNearestPoint(v:Vec) : Vec}
\operation{+ getInterSegment(p:Poly) : Segment}
\operation{+ rotate(center:Vec,angle:double) : bool}
\operation{+ move(v:Vec) : bool}
\end{class}
\begin{class}{RigidBody}{-4,13.5}
\attribute{- m\_Shape : Poly}
\attribute{- m\_Force : Vec}
\attribute{- m\_Veolocity : Vec}
\attribute{- m\_Mass : double}
\attribute{- m\_InertiaConstant : double}
\attribute{- m\_AngularVelocity : double}
\attribute{- m\_Id : int}
\attribute{- m\_CoolDown : int}
\operation{- applyForce() : void}
\operation{- move(dt:double) : void}
\operation{- accelerate(dt:double) : void}
\operation{- rotate(dt:double) : void}
\operation{- collide(Tag:RigidBody) : bool}
\end{class}
\composition{Segment}{m\_vertex1}{m\_vertex2}{Vec}
\composition{Poly}{m\_CenterPoint}{m\_Point}{Vec}
\composition{RigidBody}{m\_Shape}{}{Poly}
\end{tikzpicture}
\newpage
	
\clearpage
\begin{tikzpicture}
\begin{class}[text width=5cm]{Parameter}{0,1}
\end{class}
\begin{class}[text width=5cm]{IntParameter}{-4,3}
\inherit{Parameter}
\attribute{- val\_ : int}
\operation{+ getInt() : int}
\end{class}
\begin{class}[text width=5cm]{VecParameter}{4,3}
\inherit{Parameter}
\attribute{- v\_ : Vec}
\operation{+ getVec() : Vec}
\end{class}
\begin{class}[text width=5cm]{ShapeParameter}{0,6.5}
\inherit{Parameter}
\attribute{- id\_ : int}
\attribute{- actionMode\_ : int}
\operation{+ getId() : int}
\operation{+ getActionMode() : int}
\end{class}
\begin{class}[text width=5cm]{PolyParameter}{-4,9}
\inherit{Parameter}
\attribute{- poly\_ : Poly}
\operation{+ getPoly() : Poly}
\end{class}
\begin{class}[text width=5cm]{RigidBodyParameter}{4,9}
\inherit{Parameter}
\attribute{- rb\_ : RigidBody}
\operation{+ getRigidBody() : RigidBody}
\end{class}
\end{tikzpicture}

\section{Model Layer}
\begin{tikzpicture}
\begin{class}[text width=7cm]{PhysicsSpace}{0,-10}
\attribute{- m\_RigidBodySet : vector<RigidBody>}
\attribute{- m\_Force : Vec}
\attribute{- m\_Stepsize : double}
\operation{+ addRigidBody(InputRigidBody:RigidBody) : void}
\operation{+ deleteRigidBody(InputId:int) : void}
\operation{+ setStepSize(dt:double) : void}
\operation{+ goStep(n:int) : void}
\operation{+ addFprceField(InputForce:Vec) : void}
\operation{+ setForceField(InputForce:Vec) : void}
\operation{+ clearNonINFRigidBody() : void}
\end{class}
\begin{class}[text width=8.2cm]{Model}{0,0}
\attribute{- physicsSpace : PhysicsSpace}
\attribute{- updatePolyViewCommand : shared\_ptr<Command>}
\attribute{- refreshViewCommand : shared\_ptr<Command>}
\operation{+ createRigidBodyData(rb:RigidBody) : void}
\operation{+ adjustRigidBodyData(rb:RigidBody,id:int) : void}
\operation{+ deleteRigidBodyData(id:int) : void}
\operation{+ simulateTimeFlyData(turns:int) : void}
\operation{+ addForceFieldData(v:Vec) : void}
\operation{+ clearUserRigidBody() : void}
\operation{+ onCreatePolyView(poly:Poly,id:int) : void}
\operation{+ onAdjustPolyView(poly:Poly,id:int) : void}
\operation{+ onDeletePolyView(id:int) : void}
\operation{+ onCreateInvisiblePolyView(poly:Poly,id:int) : void}
\operation{+ onRefreshView() : void}
\end{class}
\begin{class}[text width=4cm]{RigidBody}{0,-18}
\end{class}

\composition{Model}{}{physicsSpace}{PhysicsSpace}
\composition{PhysicsSpace}{}{m\_RigidBodySet}{RigidBody}
\end{tikzpicture}
\newpage

\section{ViewModel Layer}
\begin{tikzpicture}
\begin{class}[text width = 3cm]{Model}{6,-3.2}\end{class}
\begin{class}[text width = 3cm]{View}{6,-1.6}\end{class}
\begin{class}[text width = 3cm]{Windows}{6,0}\end{class}
\begin{class}[text width = 5cm]{Parameter}{-4,-3.2}
\end{class}
\begin{class}[text width = 10cm]{Command}{-2.5,0}
\attribute{- param\_ : shared\_ptr<parameter>}
\operation{+ set\_parameters(param:shared\_ptr<Parameter>) : void}
\operation{+ pass() : virtual void}
\end{class}
\composition{Command}{param\_}{}{Parameter}
\begin{class}[text width = 7cm]{AddForceFieldDataCommand}{-4,2}
\inherit{Command}
\attribute{- viewmodel : shared\_ptr<ViewModel>}
\operation{pass() : void}
\end{class}
\begin{class}[text width = 7cm]{ClearUserRigidBodyCommand}{4,2}
\inherit{Command}
\attribute{- viewmodel : shared\_ptr<ViewModel>}
\operation{pass() : void}
\end{class}
\begin{class}[text width = 7cm]{RefreshViewCommand}{-4,4}
\inherit{Command}
\attribute{- viewmodel : shared\_ptr<ViewModel>}
\operation{pass() : void}
\end{class}
\begin{class}[text width = 7cm]{SimulateTimeFlyDataCommand}{4,4}
\inherit{Command}
\attribute{- viewmodel : shared\_ptr<ViewModel>}
\operation{pass() : void}
\end{class}
\begin{class}[text width = 7cm]{UpdatePolyViewCommand}{-4,6}
\inherit{Command}
\attribute{- viewmodel : shared\_ptr<ViewModel>}
\operation{pass() : void}
\end{class}
\begin{class}[text width = 7cm]{UpdateRigidBodyDataCommand}{4,6}
\inherit{Command}
\attribute{- viewmodel : shared\_ptr<ViewModel>}
\operation{pass() : void}
\end{class}
\begin{class}[text width = 12cm]{ViewModel}{0,15}
\inherit{View}
\inherit{Model}
\inherit{Windows}
\inherit{UpdatePolyViewCommand}
\inherit{UpdateRigidBodyDataCommand}
\inherit{RefreshViewCommand}
\inherit{SimulateTimeFlyDataCommand}
\inherit{AddForceFieldDataCommand}
\inherit{ClearUserRigidBodyCommand}
\inherit{Command}
\attribute{- model : shared\_ptr<Model>}
\attribute{- view : shared\_ptr<View>}
\attribute{- windows : shared\_ptr<Windows>}
\attribute{- updateRigidBodyDataCommand : shared\_ptr<Command>}
\attribute{- updatePolyViewCommand : shared\_ptr<Command>}
\attribute{- simulateTimeFlyDataCommand : shared\_ptr<Command>}
\attribute{- addForceFieldDataCommand : shared\_ptr<Command>}
\attribute{- refreshViewCommand : shared\_ptr<Command>}
\attribute{- clearUserRigidBodyCommand : shared\_ptr<Command>}
\operation{+ bind(model:shared\_ptr<Model>) : void}
\operation{+ bind(view:shared\_ptr<View>) : void}
\operation{+ bind(windows:shared\_ptr<Windows>) : void}
\operation{+ execUpdateRigidBodyDataCommand(rb:RigidBody,id:int,mode:int) : void}
\operation{+ execUpdatePolyViewCommand(poly:Poly,id:int,mode:int) : void}
\operation{+ execSimulateTimeFlyDataCommand(turns:int) : void}
\operation{+ execAddForceFieldDataCommand(v:Vec) : void}
\operation{+ execRefreshViewCommand() : void}
\operation{+ execClearUserRigidBodyCommand() : void}
\end{class}
\end{tikzpicture}
\newpage

\section{View Layer}
\begin{tikzpicture}
\begin{class}[text width = 5cm]{Vec}{8,0}
\end{class}
\begin{class}[text width = 5cm]{ViewPoint}{8,2.8}
\inherit{Vec}
\operation{+ getintX() : int}
\operation{+ getintY() : int}
\operation{+ getintMagnitude() : int}
\end{class}
\begin{class}[text width = 5cm]{ViewShape}{1,4}
\attribute{- m\_id : int}
\attribute{- m\_visible : bool}
\operation{+ drawShape() : virtual void}
\operation{+ setShape(id:int,v:bool) : bool}
\operation{+ getid() : int}
\operation{+ getvisible() : bool}
\end{class}
\begin{class}[text width = 8cm]{ViewPolygon}{8,9}
\inherit{ViewShape}
\attribute{- m\_pointset : vector<ViewPoint>}
\attribute{- m\_edgewidth : int}
\attribute{- m\_edgecolor : Fl\_Color}
\attribute{- m\_fillcolor : Fl\_Color}
\operation{+ setPolygon(p:Poly,id:int) : bool}
\operation{+ setPolygon(id:int,ew:int,ec:Fl\_Color,fc:Fl\_Color) : bool}
\operation{+ getedgewidth() : int}
\operation{+ getedgecolor() : Fl\_Color}
\operation{+ getfillcolor() : Fl\_Color}
\end{class}
\begin{class}[text width = 5cm]{Fl Double Window}{-1,0}
\end{class}
\begin{class}[text width = 6.5cm]{ViewWindow}{0,10.5}
\inherit{Fl Double Window}
\attribute{- m\_winvisible : bool}
\attribute{- m\_shapeset : vector<ViewShape*>}
\operation{+ getwidth() : int}
\operation{+ getheight() : int}
\operation{+ gettopleftX() : int}
\operation{+ gettopleftY() : int}
\operation{+ clearshapeset() : bool}
\operation{+ attach(vs:ViewShape*) : bool}
\operation{+ getviewshape(id:int) : ViewShape*}
\operation{+ deleteshape(vs:ViewShape*) : bool}
\operation{+ draw() : void}
\end{class}
\begin{class}[text width = 7cm]{ViewSystem}{0,17}
\attribute{- m\_windowset : vector<ViewWindow*>}
\attribute{- m\_DEFAULT\_WINDOW : ViewWindow*}
\operation{+ getwindownum() : int}
\operation{+ getWindow(name:char*) : ViewWindow*}
\operation{+ deletewindow(temp:ViewWindow*) : bool}
\operation{+ getWINDOW() : ViewWindow*}
\operation{+ setWINDOW(vw:ViewWindow*) : bool}
\operation{+ attach(vw:ViewWindow*) : bool}
\operation{+ drawSystem() : void}
\end{class}
\begin{class}[text width = 7cm]{View}{9,18}
\attribute{- m\_system : ViewSystem}
\attribute{- windows : shared\_ptr<Windows>}
\attribute{- viewPtr : shared\_ptr<View>}
\operation{+ createViewWindow(name:string) : bool}
\operation{+ createViewPolygon(id:int,p:Poly) : bool}
\operation{+ changeViewWindow(c:Fl\_Color) : bool}
\operation{+ deleteViewPolygon(id:int) : bool}
\operation{+ changeViewPolygon(id:int,p:Poly) : bool}
\operation{+ changeViewPolygon(id:int,lw:int) : bool}
\operation{+ changeViewPolygon(id:int,c:Fl\_Color,sta:bool) : bool}
\operation{+ hideViewPolygon(id:int) : bool}
\operation{+ showViewPolygon(id:int) : bool}
\operation{+ getsystem() : ViewSystem}
\operation{+ refresh() : void}
\end{class}
\begin{class}[text width = 4cm]{Windows}{9,10.3}
\end{class}
\composition{View}{m\_system}{}{ViewSystem}
\composition{ViewPolygon}{m\_pointset}{}{ViewPoint}
\composition{ViewWindow}{m\_shapeset}{}{ViewShape}
\composition{ViewSystem}{m\_DEFAULT\_WINDOW}{m\_windowset}{ViewWindow}
\composition{View}{windows}{}{Windows}
\end{tikzpicture}
\newpage

\section{Window Layer}
\begin{tikzpicture}
\begin{class}[text width = 7cm]{Parameter}{0,-5}
\end{class}
\begin{class}[text width = 10cm]{Command}{0,-1}
\attribute{- param\_ : shared\_ptr<parameter>}
\operation{+ set\_parameters(param:shared\_ptr<Parameter>) : void}
\operation{+ pass() : virtual void}
\end{class}
\composition{Command}{param\_}{}{Parameter}
\begin{class}[text width = 7cm]{AddForceFieldDataCommand}{-4,2}
\inherit{Command}
\attribute{- viewmodel : shared\_ptr<ViewModel>}
\operation{pass() : void}
\end{class}
\begin{class}[text width = 7cm]{ClearUserRigidBodyCommand}{4,2}
\inherit{Command}
\attribute{- viewmodel : shared\_ptr<ViewModel>}
\operation{pass() : void}
\end{class}
\begin{class}[text width = 7cm]{SimulateTimeFlyDataCommand}{4,4}
\inherit{Command}
\attribute{- viewmodel : shared\_ptr<ViewModel>}
\operation{pass() : void}
\end{class}
\begin{class}[text width = 7cm]{UpdateRigidBodyDataCommand}{-4,4}
\inherit{Command}
\attribute{- viewmodel : shared\_ptr<ViewModel>}
\operation{pass() : void}
\end{class}
\begin{class}[text width = 12cm]{Windows}{0,12}
\inherit{UpdateRigidBodyDataCommand}
\inherit{SimulateTimeFlyDataCommand}
\inherit{AddForceFieldDataCommand}
\inherit{ClearUserRigidBodyCommand}
\inherit{Command}
\attribute{- updateRigidBodyDataCommand : shared\_ptr<Command>}
\attribute{- simulateTimeFlyDataCommand : shared\_ptr<Command>}
\attribute{- addForceFieldDataCommand : shared\_ptr<Command>}
\attribute{- clearUserRigidBodyCommand : shared\_ptr<Command>}
\operation{+ setUpdateRigidBodyDataCommand(shared\_ptr<Command>) : void}
\operation{+ setSimulateTimeFlyDataCommand(shared\_ptr<Command>) : void}
\operation{+ setAddForceFieldDataCommand(shared\_ptr<Command>) : void}
\operation{+ setClearUserRigidBodyCommand(shared\_ptr<Command>) : void}
\operation{+ onUpdateRigidBodyData(mode:int,rb:RigidBody,id:int) : void}
\operation{+ onUpdateRigidBodyData(mode:int,id:int) : void}
\operation{+ onSimulateTimeFlyData(turns:int) : void}
\operation{+ onAddForceFieldData(v:Vec) : void}
\operation{+ onClearUserRigidBodyCommand() : void}
\end{class}
\end{tikzpicture}
\newpage

\section{App Layer}
\begin{tikzpicture}
\begin{class}[text width = 3cm]{Model}{-2,-1}\end{class}
\begin{class}[text width = 3cm]{View}{2,-1}\end{class}
\begin{class}[text width = 3cm]{Windows}{4,2}\end{class}
\begin{class}[text width = 3cm]{ViewModel}{-4,2}\end{class}
\begin{class}[text width = 8cm]{App}{0,6}
\attribute{- view : shared\_ptr<View>}
\attribute{- model : shared\_ptr<Model>}
\attribute{- windows : shared\_ptr<Windows>}
\attribute{- viewmodel : ViewModel*}
\operation{+ startWorld() : void}
\end{class}
\composition{App}{view}{}{View}
\composition{App}{model}{}{Model}
\composition{App}{windows}{}{Windows}
\composition{App}{viewmodel}{}{ViewModel}
\end{tikzpicture}
\end{document}